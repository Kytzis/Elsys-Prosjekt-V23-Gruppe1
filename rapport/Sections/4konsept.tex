
\section{Konsept}
\label{sec:konsept}
\subsection{Beskrivelse}

Løsningen ut fra innsiktsfasen, med mål om å forbedre publikumopplevelsen til NRK,
er et system kalt \Projectname. Dette systemet skal kunne monteres på briller, endten egne eller utdelte, som ved hjelp av "Heads-up-display" teknologi gir brukeren mulighet til å følge med på teksting under live arrangementer til NRK. Systemet mottar signal fra direkteteksterne i NRK, sender det på en OLED skjerm som deretter blir projisert på et glass rett foran brilleglassene. For å gjøre tekstingen minst mulig forstyrrende brukes det en linse som korregerer nødvendig fokus distanse for øye slik at det blir enklere å se på både arrangementet og teksten. 
\subsection{Eksisterende løsninger sammenlignet med \Projectname}

\epigraphfontsize{\small\itshape}
\setlength\epigraphwidth{8cm}
\setlength\epigraphrule{0pt}
\epigraphfontsize{\small\itshape}
\label{quote:1}
\epigraph{“Samtidig TEKSTING! Ja, her nærmest skriker æ for æ kunne like gjerne ha blitt hjemme for da får man med seg absolutt alt på tv ettersom man kan velge teksting eller egen tegnspråkkanal hvor de har både tekst og tegnspråk. “}{--- \textup{Tunghørt MGP fan}}


\subsection{Systemkrav}

%systemkrav tabell
\begin{table}[H]
\label{table: systemkrav}
\caption{Systemkrav. }
\begin{tabular}{|c|l|l|}
\hline
\rowcolor[HTML]{C0C0C0} 
\multicolumn{1}{|l|}{\cellcolor[HTML]{C0C0C0}Kravtype} &
  Nr. &
  Krav \\ \hline
 &
  1.1 &
  Hardwearen skal veie mindre enn TBD g. \\ \cline{2-3} 
 &
  1.2 &
  \begin{tabular}[c]{@{}l@{}}Hardwearen skal festes utenfor brukers synsfelt, \\ langs brillestangen.\end{tabular} \\ \cline{2-3} 
 &
  1.3 &
  \begin{tabular}[c]{@{}l@{}}Hardwearen skal ha max 4 cm bredde, max 8 cm lengde \\ og max 5 cm høyde\end{tabular} \\ \cline{2-3} 
\multirow{-4}{*}{Utforming} &
  1.4 &
  \begin{tabular}[c]{@{}l@{}}Systemet skal med en clips kunne festes på TBD\% \\ av alle briller\end{tabular} \\ \hline
 &
  2.1 &
  \begin{tabular}[c]{@{}l@{}}Teksten skal ha en sort bakgrunn og hvit tekst med\\ font “Sans-Serif”.\end{tabular} \\ \cline{2-3} 
\multirow{-2}{*}{Interface} &
  2.2 &
  \begin{tabular}[c]{@{}l@{}}Teksten skal oppleves å ligge mellom 1-3 meter unna \\ bruker for 8 av 10 brukere.\end{tabular} \\ \hline
 &
  3.1 &
  Systemet skal ha en batteritid på minst 3 timer. \\ \cline{2-3} 
 &
  3.2 &
  \begin{tabular}[c]{@{}l@{}}Systemet skal skrus av og på med \\ en bryter.\end{tabular} \\ \cline{2-3} 
\multirow{-3}{*}{\begin{tabular}[c]{@{}c@{}}Elektriske \\ egenskaper\end{tabular}} &
  3.3 &
  \begin{tabular}[c]{@{}l@{}}Systemet skal være TBD sekunder bak \\ direktesenderens system.\end{tabular} \\ \hline
Kommunikasjon &
  4.1 &
  \begin{tabular}[c]{@{}l@{}}Systemet skal være koblet til samme system\\  direkteteksterenes bruker over WiFi.\end{tabular} \\ \hline
\end{tabular}
\end{table}

\paragraph{Utforming: } 
Systemkrav nr. 1.1-1.4 er satt for å oppfylle brukerkrav nr. 1.1-1.2. Oppfyllelse av utformingsssytemkravene fører til at brillene unngår å sitte skjevt på hodet. Det kan også føre til systemet unngår å forstyrre opplevelsen av arrangementet. Som nevnt i systemkrav 1.4 skal systemet festes med en klips. Grunnen til det det skal være en klips er at systemet skal kunne festes på TBD\% av briller.  


\paragraph{Interface: }
Disse må revurderes...


\paragraph{Elektriske egenskaper: }
Systemkrav 3.1 er satt med bakgrunn at systemet er til å brukes i situasjoner uten tilstrekkelig strømtilkobling. Det er da avgjørende at systemet har en minimumsbatteritid på tre timer. Neste systemkrav 3.2 som er at systemet skal skrus av og på med en bryter oppfyller brukerkrav nr. 3.1. Valget om å begrense systemets funksjonalitet til en enkel av/på-bryter kan videre sees som et tiltak for å sikre en enkel og intuitiv brukeropplevelse. Systemkrav nr 3.3 er satt med bakgrunn i at systemet er ment å brukes samtidig med en direktesending. Her er det viktig å sikre at systemet ikke ligger langt bak i tid. En forsinkelse på TBD sekunder kan være akseptabelt for bruk under NRK sine livearrangementer. 


\paragraph{Kommunikasjon: }
Det siste systemkravet 4.1 som er innenfor kommunikasjon er satt på grunn av pålitelighet i WiFi. WiFi-tilkobling kan ofte være mer pålitelig enn andre former for trådløs tilkobling, og dermed gi en mer stabil og feilfri kommunikasjon mellom systemet og direkteteksteren.


