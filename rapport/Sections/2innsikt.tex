\section{Innsikt}
\label{sec:innsikt}
For å kartlegge mulige forbedringer for MGP og ulike behov publikum har under arrangementet, ble det lagt ut en undersøkelse i ulike Facebook-grupper for MGP-interesserte mennesker. Undersøkelsen ga deltakerne mulighet til å gi både positiv og konstruktiv tilbakemelding når det gjaldt arrangementet. Resultatet av undersøkelsen ga en generell oppfattelse av at blant annet sikt og informasjonsflyt var problematisk. 

Dette fikk oss til å tenke på hvordan døve og tunghørte mennesker opplever MGP, ettersom sikt blir en viktigere sans. I tillegg kan informasjonsflyten bli dårligere, da dette ofte skjer over arenaens høyttaleranlegg. 

Gjennom kontakt med en tunghørt, ble vi informert om et ønske om teksting for publikum i salen var noe som var ønskelig. Han har opplevd at man ofte velger å sitte hjemme fremfor å dra på arrangementer som MGP. Dette skyldes mye på grunn av at man lett har kontroll over teksting og tegnespråktolk når man ser på fra stuen hjemme. 

Dette ga oss en ide. Hva hvis det var mulig å ta med tekstingen på TV-skjermene inn på MGP? Ideen var at vi skulle på en måte implementere et tekstfelt "over" synet ditt, som ga muligheten om å få teksting av det som blir sagt på live arrangementer. 

For at vi skulle sjekke nytteverdien til konseptet, ble den sendt ut på Reddit, som er et onlineforum, og Facebook. Her fikk vi gode og konstruktive tilbakemelding om konseptet og hvordan dette ville blitt tatt imot av tunghørte og døve. 



Videre tanke - altså ikke ferdig skrevet :):

(DONE)- Fikk oss til å tenke på hvordan døve og mennesker opplevde MGP da både syn blir en viktigere sans. I tillegg kan informasjonsflyt bli dårligere da dette gjerne skjer gjennom tale over høyttalere. \newline
(DONE)- Kontakt med en tunghørt - skulle ønske ting var tekstet. Opplevde at det var enklere å følge med hjemmefra hvor man både kunne velge både teksting og tegnspråktolking
(DONE ish)- Kom på ideen med teksting på brilleglass\newline
- Reddit og Facebook-innlegg for å sjekke nytteverdi av ideen \newline
- Behovsanalyse?\newline
- Skal vi ha med aldersgrupper?



FACEBOOK-GRUPPER - DERSOM DET ER AV INTERESSE
Grand Prix-klubben/OGAE Norway
ESC Norge
Melodi Grand Prix. Norge + Eurovision Song Contest

\subsection{Brukerkrav}
\begin{table}[H]
\label{table: Brukerkrav}
\caption{Brukerkrav. }
\begin{tabular}{|c|l|l|}
\hline
\rowcolor[HTML]{C0C0C0} 
\multicolumn{1}{|l|}{\cellcolor[HTML]{C0C0C0}Kravtype} &
  Nr. &
  Krav \\ \hline
 &
  1.1 &
  \begin{tabular}[c]{@{}l@{}}Det skal være behagelig for bruker å ha på seg \\ systemet i minst 3 timer kontinuerlig.\end{tabular} \\ \cline{2-3} 
\multirow{-2}{*}{Utforming} &
  1.2 &
  \begin{tabular}[c]{@{}l@{}}Bruker skal kunne feste innretningen på allerede\\ eksisterende briller\end{tabular} \\ \hline
 &
  2.1 &
  \begin{tabular}[c]{@{}l@{}}Bruker skal kunne lese teksten tydelig med ulike\\ bakgrunner\end{tabular} \\ \cline{2-3} 
\multirow{-2}{*}{Interface} &
  2.2 &
  \begin{tabular}[c]{@{}l@{}}Bruker skal kunne lese teksten samtidig som bruker\\ kan se sceneshowet\end{tabular} \\ \hline
\begin{tabular}[c]{@{}c@{}}Elektriske\\ egenskaper\end{tabular} &
  3.1 &
  Bruker skal kun trenge å skru på produktet med en bryter. \\ \hline
\end{tabular}
\end{table}


\subsection{Systemkrav}

\begin{table}[H]
\label{table: systemkrav}
\caption{Systemkrav. }
\begin{tabular}{|c|l|l|}
\hline
\rowcolor[HTML]{C0C0C0} 
\multicolumn{1}{|l|}{\cellcolor[HTML]{C0C0C0}Kravtype} &
  Nr. &
  Krav \\ \hline
 &
  1.1 &
  Hardwearen skal veie mindre enn TBD g. \\ \cline{2-3} 
 &
  1.2 &
  \begin{tabular}[c]{@{}l@{}}Hardwearen skal festes utenfor brukers synsfelt, \\ langs brillestangen.\end{tabular} \\ \cline{2-3} 
 &
  1.3 &
  \begin{tabular}[c]{@{}l@{}}Hardwearen skal ha max 4 cm bredde, max 8 cm lengde \\ og max 5 cm høyde\end{tabular} \\ \cline{2-3} 
\multirow{-4}{*}{Utforming} &
  1.4 &
  \begin{tabular}[c]{@{}l@{}}Systemet skal med en clips kunne festes på TBD\% \\ av alle briller\end{tabular} \\ \hline
 &
  2.1 &
  \begin{tabular}[c]{@{}l@{}}Teksten skal ha en sort bakgrunn og hvit tekst med\\ font “Sans-Serif”.\end{tabular} \\ \cline{2-3} 
\multirow{-2}{*}{Interface} &
  2.2 &
  \begin{tabular}[c]{@{}l@{}}Teksten skal oppleves å ligge mellom 1-3 meter unna \\ bruker for 8 av 10 brukere.\end{tabular} \\ \hline
 &
  3.1 &
  Systemet skal ha en batteritid på minst 3 timer. \\ \cline{2-3} 
 &
  3.2 &
  \begin{tabular}[c]{@{}l@{}}Systemet burde kunne skrus av og på med \\ en bryter.\end{tabular} \\ \cline{2-3} 
\multirow{-3}{*}{\begin{tabular}[c]{@{}c@{}}Elektriske \\ egenskaper\end{tabular}} &
  3.3 &
  \begin{tabular}[c]{@{}l@{}}Systemet skal være TBD sekunder bak \\ direktesenderens system.\end{tabular} \\ \hline
Kommunikasjon &
  4.1 &
  \begin{tabular}[c]{@{}l@{}}Systemet skal være koblet til samme system\\  direkteteksterenes bruker over WiFi.\end{tabular} \\ \hline
\end{tabular}
\end{table}

\paragraph{Utforming: } 
Systemkrav nr. 1.1-1.4 er satt for å oppfylle brukerkrav nr. 1.1-1.2. Oppfyllelse av utformingsssytemkravene fører til at brillene unngår å sitte skjevt på hodet. Det kan også føre til systemet unngår å forstyrre opplevelsen av arrangementet. Som nevnt i systemkrav 1.4 skal systemet festes med en klips. Grunnen til det det skal være en klips er at systemet skal kunne festes på TBD\% av briller.  


\paragraph{Interface: }
Systemkrav nr. 2.1-2.2 er satt med bakgrunn i 


\paragraph{Elektriske egenskaper: }
\paragraph{Kommunikasjon: }




